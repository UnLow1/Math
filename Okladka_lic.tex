\documentclass[12pt,a4paper]{article}
\usepackage{polski}
\usepackage{wmstitle_lic}
\usepackage{graphicx}
\usepackage{color}
\usepackage{mathtools}
\usepackage{float}
\usepackage{mathrsfs}

\newtheorem{df}{Definicja}[section]
\newtheorem{pr}{Przyk{\l}ad}[section]

\begin{document}

\title{Hipoteza \v Cernego}
\author{Sylwia Klimkiewicz, Adam Jamka}
\promotor{Wit Fory\'s}
\nralbumu{276782, 276765}
\maketitle


%% SEKCJA 1 Podstawowe definicje
\section{Podstawowe definicje.}
%% def 1.1
\begin{df} 
Grafem nazywamy \textbf{G}=(\textbf{V},\textbf{E}), gdzie \textbf{V} jest niepustym zbiorem, kt\'orego elementy zwane s\k{a} wierzcho{\l}kami a \textbf{E} jest rodzin\k{a} dwuelemntowych podzbior\'ow zbioru \textbf{V}, zwanych kraw\k{e}dziami.
$\textbf{E}=\{vw : v,w\in\textbf{V}\}$.
\end{df}
%% def 1.2
\begin{df} 
Niech \textbf{G}=(\textbf{V},\textbf{E}) b\k{e}dzie grafem. Liczb\k{e} wierzcho{\l}\'ow grafu \textbf{G} nazywamy rz\k{e}dem grafu i oznaczamy $|\textbf{V}|$, natomiast liczb\k{e} kraw\k{e}dzi nazywamy rozmiarem grafu i oznaczamy $|\textbf{E}|$
\end{df}
%% def 1.3
\begin{df} 
Niech \textbf{G}=(\textbf{V},\textbf{E}) b\k{e}dzie grafem. Zbi\'or s\k{a}siad\'ow wierzcho{\l}ka $v\in\textbf{V}$ $\textbf{N}(v)$ sk{\l}ada si\k{e} z wszystkich wierzcho{\l}k\'ow grafu \textbf{G}, takich \.ze istniej\k{a} kraw\k{e}dzie nale\.z\k{a}ce do zbioru \textbf{E}, {\l}\k{a}cz\k{a}ce te wierzcho{\l}ki z v. $\textbf{N}(v)=\{w : vw\in\textbf{E}\}$.
\end{df} 
%% def 1.4
\begin{df} 
Niech \textbf{G}=(\textbf{V},\textbf{E}) b\k{e}dzie grafem. Stopniem wierzcho{\l}ka $v\in\textbf{V}$ nazywamy liczb\k{e} jego s\k{a}siad\'ow i oznaczamy $deg(v)$.
\end{df}
%% def 1.5
\begin{df} 
Graf \textbf{G}=(\textbf{V},\textbf{E}) jest r-regularny, je\.zeli wszystkie jego wierzcho{\l}ki maj\k{a} stopie\'n r\'owny r.
\end{df} 
%% def 1.6
\begin{df} 
Kraw\k{e}dzi\k{a} skierowan\k{a} lub {\l}ukiem grafu \textbf{G}=(\textbf{V},\textbf{E}) nazywamy uporz\k{a}dkowan\k{a} par\k{e} wierzcho{\l}k\'ow $e=(v,w)$, gdzie $v\in\textbf{V}$ jest pocz\k{a}tkiem {\l}uku $e$, natomiast $w\in\textbf{V}$ jego ko\'ncem. Graf sk{\l}adaj\k{a}cy si\k{e} z kraw\k{e}dzi skierowanych nazywamy grafem skierowanym.
\end{df}
%% def 1.7
\begin{df} 
Niech \textbf{G}=(\textbf{V},\textbf{E}) b\k{e}dzie grafem. Ci\k{a}g wierzcho{\l}k\'ow $(v_{1},\ldots,v_{n})$, taki \.ze $\forall_{i}$ $v_{i}\in\textbf{V}$ oraz  $(v_{i-1},v_{i})\in\textbf{E}$ dla $i=2,\ldots,n$ nazywamy \'scie\.zk\k{a} w grafie \textbf{G}.
\end{df} 
%% def 1.8
\begin{df} 
Graf \textbf{G} jest sp\'ojny je\'sli dowolne dwa jego wierzcho{\l}ki s\k{a} po{\l}\k{a}czone \'scie\.zk\k{a}.
\end{df}
%% def 1.9
\begin{df} 
Automatem nazywamy tr\'{o}jk\k{e} $\mathscr{A}=(Q, \Sigma, \delta)$, gdzie Q jest zbiorem mo\.{z}liwych stan\'{o}w, $\Sigma$ jest alfabetem, natomiast $\delta:Q x \Sigma \rightarrow$ Q jest funkcj\k{a} definiuj\k{a}c\k{a} 	zachowanie si\k{e} litery nale\.{z}\k{a}cej do alfabetu $\Sigma$ w stanie Q.
\end{df}


%% SEKCJA 2 Wprowadzenie do problemu
\newpage
\section{Wprowadzenie do problemu. Przyk{\l}ady zastosowa\'{n}.}

%%Przykład 2.1
\begin{pr}
Przyk{\l}adem synchronicznego automatu z czterema stanami Q = \textbraceleft 1, 2, 3, 4\textbraceright  oraz alfabetem sk{\l}adaj\k{a}cym si\k{e} z dw\'{o}ch liter $\Sigma$ = \textbraceleft a, b\textbraceright  mo\.{z}e by\'c poni\.{z}szy graf.
%% Rysunek 1
\begin{figure}[H]
    \centering
    \includegraphics[width=0.62\textwidth]{rysunek1}
    \caption{Synchroniczny automat}
    \label{fig:rysunek1}
\end{figure}

{\L}atwo mo\.{z}emy sprawdzi\'{c}, \.{z}e s{\l}owem resetuj\k{a}cym automat na rysunku \ref{fig:rysunek1} jest \textit{abbbabbba}, co r\'{o}wnowa\.{z}nie mo\.{z}emy zapisa\'{c} \textit{a$b^{3}$a$b^{3}$a}. Pos{\l}ugujac si\k{e} tym s{\l}owem zawsze \'{s}cie\.{z}ka sko\'{n}czy si\k{e} na wierzcho{\l}ku drugim.\\
\\
\textbf{1}$\xrightarrow{a}$2$\xrightarrow{b}$3$\xrightarrow{b}$4$\xrightarrow{b}$1$\xrightarrow{a}$2$\xrightarrow{b}$3$\xrightarrow{b}$4$\xrightarrow{b}$1$\xrightarrow{a}$\textbf{2}\\
\textbf{2}$\xrightarrow{a}$2$\xrightarrow{b}$3$\xrightarrow{b}$4$\xrightarrow{b}$1$\xrightarrow{a}$2$\xrightarrow{b}$3$\xrightarrow{b}$4$\xrightarrow{b}$1$\xrightarrow{a}$\textbf{2}\\
\textbf{3}$\xrightarrow{a}$3$\xrightarrow{b}$4$\xrightarrow{b}$1$\xrightarrow{b}$2$\xrightarrow{a}$2$\xrightarrow{b}$3$\xrightarrow{b}$4$\xrightarrow{b}$1$\xrightarrow{a}$\textbf{2}\\
\textbf{4}$\xrightarrow{a}$4$\xrightarrow{b}$1$\xrightarrow{b}$2$\xrightarrow{b}$3$\xrightarrow{a}$3$\xrightarrow{b}$4$\xrightarrow{b}$1$\xrightarrow{b}$2$\xrightarrow{a}$\textbf{2}
\end{pr}

Rozwa\.{z}my teraz troszk\k{e} bardziej rozbudowany automat, tak zwany automat Ashby. Przyk{\l}ad ten opisuje jak poradzi\'{c} sobie ze st{\l}umieniem ha{\l}asu generowanego przez dwa \'{z}r\'{o}d{\l}a - \'{s}piew oraz \'{s}miech.

%% Przykład 2.2
\begin{pr}
W tym automacie mamy tyle samo stan\'{o}w co wcze\'{s}niej  Q = \textbraceleft 00, 01, 10, 11\textbraceright, lecz ilo\'{s}\'{c} liter w alfabecie zosta{\l}a zwi\k{e}kszona do czterech $\Sigma$ = \textbraceleft a, b, c, d\textbraceright. Ka\.{z}dy stan sk{\l}ada si\k{e} z dw\'{o}ch cyfr, pierwsza b\k{e}dzie odpowiada{\l}a za ha{\l}as generowany przez \'{s}piew, a druga przez \'{s}miech, gdzie \textit{0} oznacza\'{c} b\k{e}dzie brak ha{\l}asu, natomiast \textit{1} generowanie nieprzyjemnego d\'{z}wi\k{e}ku.

%% Rysunek 2
\begin{figure}[H]
    \centering
    \includegraphics[width=1.05\textwidth]{rysunek2}
    \caption{Automat Ashby}
    \label{fig:rysunek2}
\end{figure}

Rozwa\.{z}aj\k{a}c s{\l}owo \textit{acb} szybko stwierdzimy, \.{z}e jest ono s{\l}owem resetuj\k{a}cym automat przedstawiony na Rysunku \ref{fig:rysunek2}.\\
\\
\textbf{00}$\xrightarrow{a}$01$\xrightarrow{c}$01$\xrightarrow{b}$\textbf{00}\\
\textbf{01}$\xrightarrow{a}$01$\xrightarrow{c}$01$\xrightarrow{b}$\textbf{00}\\
\textbf{10}$\xrightarrow{a}$10$\xrightarrow{c}$00$\xrightarrow{b}$\textbf{00}\\
\textbf{11}$\xrightarrow{a}$01$\xrightarrow{c}$00$\xrightarrow{b}$\textbf{00}\\

Oznacza to, i\.{z} po zastosowaniu tego s{\l}owa ha{\l}as nie b\k{e}dzie generowany ani przez \'{s}piew, ani przez \'{s}miech, a wi\k{e}c b\k{e}dziemy w stanie do kt\'{o}rego d\k{a}\.{z}yli\'{s}my.
\end{pr}

Po przeanalizowaniu powy\.{z}szych przyk{\l}ad\'{o}w dochodzimy do wniosku, i\.{z} po zastosowaniu s{\l}owa resetuj\k{a}cego ko\'{n}czymy prac\k{e} w znanym z g\'{o}ry stanie, lecz nie wiemy z kt\'{o}rego stanu zaczynali\'{s}my.

Nast\k{e}pny przyk{\l}ad poka\.{z}e jakie zastosowanie mo\.{z}e dany problem mie\'{c} w dziedzinie przemys{\l}u, handlu, gdzie np. b\k{e}dziemy pakowa\'{c} jakie\'{s} elementy, kt\'{o}re maj\k{a} okre\'{s}lony kszta{\l}t.

%% Przykład 2.3
\begin{pr}
Przypu\'{s}\'{c}my, \.{z}e nasz przedmiot ma kszta{\l}t wielok\k{a}ta przedstawionego na Rysunku \ref{fig:rysunek3}.

%% Rysunek 3
\begin{figure}[H]
    \centering
    \includegraphics[width=0.23\textwidth]{rysunek3}
    \caption{Wielok\k{a}tny element}
    \label{fig:rysunek3}
\end{figure}


Ka\.{z}dy taki element wk{\l}adamy do pude{\l}ka, lecz zanim to zrobimy musimy je odpowiednio posortowa\'{c}, tak aby mia{\l}y t\k{a} sam\k{a} orientacj\k{e}. Dla u{\l}atwienia przypu\'{s}my, \.{z}e s\k{a} mo\.{z}liwe tylko cztery pozycje tego elementu, kt\'{o}re s\k{a} widoczne na Rysunku \ref{fig:rysunek4}.

%% Rysunek 4
\begin{figure}[H]
    \centering
    \includegraphics[width=0.87\textwidth]{rysunek4}
    \caption{Mo\.{z}liwe orientacje elementu}
    \label{fig:rysunek4}
\end{figure}

Za{\l}\'{o}\.{z}my, \.{z}e interesuj\k{a}ca dla nas jest druga od lewej strony orientacja elementu. Skorzystamy z lekko zmodyfikowanego automatu, kt\'{o}ry juz wcze\'{s}niej poznali\'{s}my, a dok{\l}adniej z automatu przedstawionego na Rysunku \ref{fig:rysunek1}. G{\l}\'{o}wna zmiana b\k{e}dzie polega{\l}a na tym, i\.{z} moz\.{z}liwe stany, kt\'{o}re automat mo\.{z}e przyj\k{a}\'{c} b\k{e}d\k{a} kolejnymi orientacjami tego elementu.
\\
%% Rysunek 5
\begin{figure}[H]
    \centering
    \includegraphics[width=0.7\textwidth]{rysunek5}
    \caption{Zmiana orientacji elementu}
    \label{fig:rysunek5}
\end{figure}

Pami\k{e}taj\k{a}c s{\l}owo resetuj\k{a}ce w automacie z Rysunku \ref{fig:rysunek1} abbbabbba wnioskujemy, \.{z}e s{\l}owem resetuj\k{a}cym automat na Rysunku \ref{fig:rysunek5} jest low-HIGH-HIGH-HIGH-low-HIGH-HIGH-HIGH-low, a stan ko\'{n}cowy to orientacja elementu, kt\'{o}r\k{a} chcieli\'{s}my uzyska\'{c}.
\end{pr}

Jak wida\'{c} nawet dla ma{\l}o skomplikowanego automatu synchronicznego mo\.{z}emy znale\'{z}\'{c} proste zastosowanie, kt\'{o}re oczywi\'{s}cie mo\.{z}emy uog\'{o}lnia\'{c}, a sam automat stara\'{c} si\k{e} rozwija\'{c}, powi\k{e}ksza\'{c}.


%% SEKCJA 2 WKW na to, żeby graf był synchronizowalny
\newpage
\section{Warunek konieczny i wystarczaj\k{a}cy na synchronizowalno\'{s}\'{c} automatu.}

Oczywistym faktem jest, \.{z}e nie ka\.{z}dy sko\'{n}czony automat jest synchroniczny. Nasuwa si\k{e} wi\k{e}c pytanie, je\'{s}li mamy dany sko\'{n}czony automat $\mathscr{A}=(Q, \Sigma, \delta)$, jak stwierdzi\'{c} czy jest on synchroniczny czy nie?

Konstruujemy automat mocy $P(\mathscr{A})$. Jego zbi\'{o}r stan\'{o}w to zbi\'{o}r $P'(Q)$, kt\'{o}ry jest niepustym podzbiorem Q, a funkcja przejscia $\delta$ jest naturalnie rozszerzona do zbioru $P'(Q)x\Sigma$. Inaczej m\'{o}wi\k{a}c dla ka\.{z}dego niepustego podzbioru $P$ zbioru $Q$ oraz $a \epsilon \Sigma$ mamy $\delta(P,a)=\{\delta(p,a) | p\epsilon P\}$.
\\
%% Rysunek 6
\begin{figure}[H]
    \includegraphics[width=1.1\textwidth]{rysunek6}
    \caption{Automat mocy}
    \label{fig:rysunek6}
\end{figure}

Na Rysunku \ref{fig:rysunek6} przedstawiony zosta{\l} automat mocy dla sko\'{n}czonego automatu przedstawionego na Rysunku \ref{fig:rysunek1}. S{\l}owo $w \epsilon \Sigma$ jest s{\l}owem resetuj\k{a}cym automat $\mathscr{A}$ je\'{s}li w $P(\mathscr{A})$ \'{s}cie\.{z}ka wygenerowana przez to s{\l}owo, zaczynaj\k{a}ca si\k{e} w dowolnym stanie $q \epsilon Q$ ko\'{n}czy si\k{e} zawsze w tym samym wierzcho{\l}ku. W naszym przypadku s{\l}owem resetuj\k{a}cym jest ju\.{z} wcze\'{s}niej znalezione s{\l}owo $ab^3ab^3a$, a wierzcho{\l}kiem ko\'{n}cowym - wierzcho{\l}ek $2$. Ponadto jest to najkr\'{o}tsze s{\l}owo resetuj\k{a}ce. 

Wnioskujemy wi\k{e}c, \.{z}e algorytm szukania s{\l}owa resetuj\k{a}cego automat $\mathscr{A}$ nie jest skomplikowany. Najpierw tworzymy automat mocy $P(\mathscr{A})$, nast\k{e}pnie szukamu s{\l}owa, dzi\k{e}ki kt\'{o}remu startuj\k{a}c z dowolnego stanu dostaniemy si\k{e} do jednego, tego samego wierzcho{\l}ka - mo\.{z}emy tutaj zastosowa\'{c} algorytm przeszukiwania 
wszerz (BFS), kt\'{o}ry pomo\.{z}e znale\'{z}\'{c} takie s{\l}owo albo stwierdzi\'{c}, i\.{z} ono nie istenieje. Problem jednak w tym, \.{z}e rozmiar automatu mocy $P(\mathscr{A})$ jest wyk{\l}adniczo wi\k{e}kszy od rozmiaru automatu $\mathscr{A}$. Otrzymany w ten spos\'{o}b algorytm ma z{\l}o\.{z}ono\'{s}\'{c} wielomianow\k{a}.



\end{document}
